\chapter{Conclusion}

A non-destructive method was used to monitor the effect of thermal cycling on the thermal conductivity and delamination of two thermal adhesives.  By testing on a standardized, application-specific, package, the data obtained can be used to make design decisions with confidence that the results are indicative of real world performance.

It was found that the clamped PSA, unclamped PSA, and the clamped alternative adhesive showed an increase in thermal performance with cycling, possibly due to curing, which supports earlier findings in the literature.  There were no signs of delamination in the PSA after 1000 cycles.  However, delamination was detected at 17 out of the 35 locations tested in the boards attached with the alternative adhesive.  

While the method used is not capable of detecting the presence of delamination in overmolded boards, it was seen that the thermal coductivity of the PSA adhesive showed similar improvements due to curing with the overmolding as it did without the overmolding.  It was also found that severe cracking occurred in overmolding A after 250 cycles while no cracking occurred in overmolding B after 1000 cycles.

volume fraction, specific heat, and thermal conductivity of an epoxy impregnated with alumina granules were measured.  The properties were found to be non-uniform both spatially and temporally. 

The volume fraction of alumina was measured using optical microscopy and image processing.  It was found to vary by location, both axially and radially.  However, even the lowest volume fractions measured were still higher than the statistically predicted volume fraction for spherical particles \cite{McGeary1961}.  This is attributed to the process used to combine the epoxy and alumina granules and the range of particle sizes in the mixture.

The specific heat was measured with a DSC and was found to vary by location, ranging from \(0.86\textrm{--}1.14\:\mathrm{J/g\cdot K}\) at 25\(^{\circ}\mathrm{C}\) and \(1.07\textrm{--}1.37\:\mathrm{J/g\cdot K}\) at \(125^{\circ}\mathrm{C}\).  This is consistent with the conclusion of other authors \cite{McGrath2008, Wong1999, Mamunya2002, Singh2003} that the thermal properties are strongly dependent on volume fraction.  

The apparent thermal conductivity was measured using a guarded heater method based on the ASTM D5470-06 and was found to range from \(1.96\:\mathrm{W/m\cdot K}\) at \(25^{\circ}\mathrm{C}\) to \(3.17\:\mathrm{W/m\cdot K}\) at \(100^{\circ}\mathrm{C}\).  Since only three samples were tested, the authors hope to extend this study to a larger, more statistically significant, dataset in the future.

There were changes in thermal properties and coloration with exposure to temperatures above \(60^{\circ}\mathrm{C}\).  This is attributed to the epoxy not being fully cured as delivered by the vendor.  Since these temperatures are within the operating range of many electronics, care needs to taken to ensure that the potting compound is fully cured prior to deployment.  Otherwise, the properties of the potting compound could change drastically and non-uniformly over the life of the product, leading to unpredictable behavior.

Epoxy with embedded alumina granules can be used to improve the thermal performance and reliability of electronics in harsh environments.  However, users of such compounds should be aware that the thermal properties are not necessarily constant in time or uniform, and assuming that they are could lead to significant errors when modeling their performance.


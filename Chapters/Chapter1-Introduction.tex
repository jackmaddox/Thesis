\chapter{Introduction}

Ceramic oxide filled polymers are being increasingly used in the area of electronics packaging.  The motivation for this is that the composite retains the electrical and thermal properties of the ceramic while keeping the mechanical properties of the polymer.  This allows the compound to be easily incorporated into the manufacturing process used to package electronics.  One common polymer that is used is epoxy.  Epoxy has the added benefit that by using different components and curing agents its final properties can be adjusted as desired.  

A number of studies have looked at the effects of ceramic and metal powders in polymers.  It has been shown that the mechanical properties \cite{McGrath2008, Wong1999}, electrical conductivity \cite{Mamunya2002}, dielectric constant \cite{Singh2003}, and thermal conductivity \cite{Wong1999, Mamunya:2002} of metal and ceramic polymer composites show a beneficial increase with an increasing volume fraction of the embedded component.  These attributes make it a good potting compound for harsh environment electronics.

Epoxy containing alumina granules is frequently used to encapsulate magnetic devices since it has a relatively high dielectric constant and thermal conductivity while still being conformal enough to be easily applied during assembly.  The higher thermal conductivity of the potting compound helps to remove the heat generated by the device.  The mechanical support supplied by the hardened resin helps to protect the device, especially in harsh environments with high shock and vibration.


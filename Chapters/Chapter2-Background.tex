\chapter{Background}

The difference in coefficients of thermal expansion (CTE) between dissimilar materials causes shear stresses to be generated during normal operation of an electronic device.  The cyclic nature of these stresses causes fatigue at the joints, which leads to cracking and ultimately failure.  Minimizing the temperature change experienced by a device will diminish the effects of CTE stresses and thus lead to improved reliability.  Therefore, the removal of heat has become an increasingly important aspect of electronics packaging in recent years.  

The heat generated in an electronic device must eventually be rejected to the environment.  This process is enhanced in a number of ways, usually with a heat sink of some kind being attached to the heat producing components in a device.  However, the thermal resistance between the components and the heat sink can become a bottleneck if measures are not taken to decrease the contact resistance between the two.  While the contact resistance could be decreased by pressing one surface against the other with sufficient force, this is not optimal from a reliability or manufacturing point of view.  Another alternative is to use a thermal interface material (TIM).  Some TIMs in use today are soft metals, thermal greases, epoxy resins, pressure sensitive adhesive tapes (PSA), phase change materials, and elastomer pads \cite{Gwinn2003,Nakayama2003}.  The properties of these materials are often enhanced by the addition of judiciously selected particles to increase the bulk thermal conductivity of the material.

The use of a thermal adhesive as a TIM gives the added benefit of securing the two surfaces together without the need for external clamping.  This reduces the complexity of a device by reducing the number of components and simplifies the manufacturing, as adhesives are relatively easy to apply when compared to the alternatives.  Adhesives, both electrically conductive and thermally conductive, are used in the assembly of many automotive electronics \cite{Murray2003}.  When used as a TIM in these applications, it is necessary that the adhesive exhibit reliable performance both mechanically and thermally.  The effects of aging \cite{Murray2003a} and thermal cycling \cite{Tuhus1993,Kwon2005,Eveloy2004,Eyman1997,Khuu2007} on the mechanical strength of various adhesives have been well documented in the literature.  However, there is less information about the effects of aging and thermal cycling on the thermal performance of these adhesives.  

This paper will compare the thermal performance of two adhesives, PSA and a new alternative adhesive.  PSA tapes are used in a variety of applications from everyday consumer products to, more recently, electronics assembly \cite{Creton2003}.  In the area of electronics assembly, PSA tapes have been developed for electrical insulation, electrical conduction, and thermal conduction \cite{Conner1995,Klosterman1998}.  Of interest, in this paper, are the thermally conductive PSAs and their use as a TIM for connecting a printed circuit board (PCB) to an aluminum plate.  The modes of failure due to separation at a polymer-metal interface were reported by Yao, et.~al.~\cite{Yao2002}, to be either adhesive failure or cohesive failure.  Adhesive failure is when the separation occurs at the interface between the adhesive and the metal, while cohesive failure is when the separation occurs within the adhesive layer.  It was reported that the mode of failure would depend primarily on the surface roughness of the metal.  The effects of creep on the mechanical strength of PSA for heat sink attachment under isothermal conditions were reported by Eveloy, et.~al.~\cite{Eveloy2004}.  This paper will look at the effects of thermal cycling on the thermal resistance of PSA attached to an aluminum substrate.  The measurement method used in this paper will not distinguish between adhesive and cohesive failure, but rather the failure of the bond due to separation will be generically referred to as delamination.  At the conclusion of the study, a destructive analysis of the boards will be peformed to determine the location of the failure.

Similar work to that being presented in this paper has been done by Eyman, et.~al.~\cite{Eyman1997} and Khuu, et.~al.~\cite{Khuu2007}.  Eyman, et.~al.~\cite{Eyman1997} tested the effects of thermal cycling on the reliability of various methods for attaching heat sinks to plastic ball grid arrays (PBGA), with PSA and heat cured epoxy being among the methods tested.  Electrical continuity was used as the criterion for failure.  Khuu, et.~al.~\cite{Khuu2007} used the laser flash method to measure the thermal performance of various thermal interface materials, with epoxy and gap pads attached with PSA among the tested materials.  

The selection of a TIM depends on many factors, which need to be analyzed differently for each specific application.  Since the overall thermal resistance of a TIM includes the thermal resistance of the TIM itself as well as the interface resistances between the TIM and the surfaces it touches, the bulk thermal conductivity is insufficient to fully characterize a TIM.  The inadequacies of ASTM standards for measuring the performance of TIMs, as detailed by Lasance \cite{Lasance2003}, make design decisions based on vendor-supplied data difficult.  The data supplied by vendors are usually obtained with a standard pressure of 3 MPa being applied to the sample, which is much higher than what the TIM will experience in a real world application.  It has been shown that contact resistance is greatly affected by changes in pressure \cite{Nakayama2003}.  Therefore, data collected with an unrealistically high pressure being applied cannot be reliably used to make design decisions for applications where the TIM will have a much lower pressure applied.  

As suggested by Lasance \cite{Lasance2003}, one possible solution to this problem is to implement application-specific tests.  By applying various TIMs to a standardized package, meaningful comparisons can be made between the products with confidence that the resulting data is indicative of real world performance.  

This study was conducted using a standardized package consisting of either a Flame Retardent 4 (FR4) or polyimide (denoted as Flex) PCB attached to an aluminum substrate with a thermal adhesive.  The PCBs were attached to the aluminum substrate by the vendor to control for possible inconsistencies in manufacturing procedure.  The two adhesives tested were PSA and an alternative adhesive (denoted as ALT).  The boards were rapidly aged by thermally cycling from -40 to 125$^{\circ}\mathrm{C}$.  The composite thermal resistance of the PCB, TIM, and aluminum substrate was checked every 250 cycles using a procedure similar to the one presented by Knight, et.~al.~\cite{Knight2007}.  

This study will also look at the effects of overmolding on the thermal performance of PSA.  The overmoldings used are polyimide based and ecapsulate the entire PCB and most of the aluminum substrate. Since it wraps around the PCB and substrate, the overmolding has the effect of mechanically holding the PCB in place.  This will help to minimize the effects of delamination, however there is a trade off, as it inhibits the transfer of heat from the board.  Two overmoldings were tested and they will be denoted as overmolding A and overmolding B.

While this study only measures the effects of thermal cycling on the thermal performance of the TIMs, related studies using the same standardized package have characterized the effects of thermal cycling on the reliability of the packages used to populate the board with PSA as the TIM \cite{Lall2006,Evans2008}.

An epoxy, Stycast W-19, impregnated with tabular alumina, T64, is used as a potting compound for a power transformer.  The alumina granules are packed around the device to be potted and the structure is subjected to sonic vibrations as the epoxy resin is allowed to fill in the empty spaces around the alumina. The vibration and the incorporation of a range of particle sizes result in an alumina volume fraction of approximately 70-80\%, as will be reported later.  This is higher than the expected maximum packing density for spherical particles of 64\% \cite{McGeary1961}, it is this high volume fraction of alumina that makes this composite useful as a potting compound.  

In order to model and predict the performance of the system, it is necessary to have accurate values for the properties of the material.  It is difficult to accurately predict the volume fraction, or packing density, of the alumina as well as the thermal properties.  Therefore, the properties of the material will be determined experimentally so that it can be properly modeled.


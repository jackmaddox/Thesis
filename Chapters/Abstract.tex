A non-destructive method was used to determine the effects of thermal cycling on the thermal performance of a PCB attached to an aluminum substrate with a thermal adhesive.  This method allows for a comparison of the thermal performance of various TIMs in an industrial application.  \newline
\indent Testing was done on FR4 and Flex boards, both with and without overmolding, attached using PSA and an alternative adhesive.  Baseline measurements were taken, then the boards were cycled from -40 to 125$^{\circ}$C on a 90-minute cycle with 15-minute dwells at the target temperatures. It was found that both adhesives showed an increase in thermal conductivity, possibly due to curing, and delamination occurred at 17 out of 35 locations with the alternative adhesive within the first 1000 cycles while no delamination occurred with the PSA. 

In harsh environments, including high shock and vibration, magnetic devices such as transformer coils are potted to enhance thermal performance and provide mechanical protection.  One potting compound frequently used is epoxy containing alumina particles. A nominally isotropic and uniform potting compound consisting of about 70 to 80\% by volume 14-28 mesh (0.6 to 1.2 mm across) alumina granules in low viscosity epoxy was tested to determine its thermal properties.  Examination by optical microscopy revealed that there was significant variation in volume fraction of alumina particles by location.  The specific heat and thermal conductivity of the compound were measured using a Differential Scanning Calorimeter and guarded heater method based on the ASTM D5470-06.  The thermal properties were found to vary with time, location, and temperature; with the specific heat ranging from \(1.00\:\mathrm{J/g^{\circ}\mathrm{C}}\pm 14\%\)at \(25^{\circ}\mathrm{C}\) to \(1.22\:\mathrm{J/g^{\circ}\mathrm{C}}\pm12\%\)at \(125^{\circ}\mathrm{C}\)and an apparent thermal conductivity of\(2.56\:\mathrm{W/m\cdot K}\pm 23\%\).  Users of such compounds should be aware that the thermal properties are not necessarily constant in time or uniform, and assuming that they are could lead to significant errors when modeling their performance.

	In harsh environments, including high shock and vibration, magnetic devices such as transformer coils are potted to enhance thermal performance and provide mechanical protection.  One potting compound frequently used is epoxy containing alumina particles.
	A nominally isotropic and uniform potting compound consisting of about 70 to 80\% by volume 14-28 mesh (0.6 to 1.2 mm across) alumina granules in low viscosity epoxy was tested to determine its thermal properties.  A large block, nominally 200 mm by 100 mm, of the potting compound was cast and samples were taken from various locations throughout: near the top, bottom, sides and middle.  Examination by optical microscopy revealed that there was significant variation in volume fraction of alumina particles from sample to sample.  Small alumina particles settled amongst the larger particles near the bottom of the cast block.  Near the top of the cast block, only large particles were present, so the compound was proportionally more epoxy.
	The specific heat of the compound was measured using a Differential Scanning Calorimeter.  It was noted that the specific heat of the samples changed significantly during testing.  This was apparently due to the large block sample not having been completely cured before testing.  After additional curing of the samples, the properties stopped changing.  The specific heat was found to vary with location, with the value depending on the relative abundance of alumina granules.
	The thermal conductivity of samples from various regions was measured using a guarded heater method based on ASTM D5470-06.  As with specific heat, the thermal conductivity was found to vary significantly with location, based on the relative abundance of alumina granules.
	Values are presented for the thermal properties and their relationship to alumina volume fraction is noted.  Users of such compounds should be aware that the thermal properties are not necessarily constant in time or uniform, and assuming that they are could lead to significant errors when modeling their performance.

